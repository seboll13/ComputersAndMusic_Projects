\documentclass{beamer}
%Information to be included in the title page:
\title{COM-418: Spatial Audio Project}
\author{Sebastien Ollquist, Antonio Pisanello, Daniel Suter.}
\institute{EPFL}
\date{Spring 2022}

\begin{document}
\frame{\titlepage}

\begin{frame}
    \frametitle{Project Goal}
    \begin{enumerate}
        \item Create an immersive listening experience by spatialising some music file
        \item Add sound source locations
        \item Possibly play with some filters to create a high quality listening experience
    \end{enumerate}
\end{frame}

\begin{frame}
    \frametitle{Motivations}
    \begin{enumerate}
        \item Sebastien is impressed by how this spatial technology works (he was the main reason why we chose the project to be honest, otherwise we would have done something else). Just kidding :)
        \item TODO: change this.
    \end{enumerate}
\end{frame}

\begin{frame}
    \frametitle{Challenges}
    \begin{enumerate}
        \item Understand the math behind HRTFs, binaural rendering and all that stuff
    \end{enumerate}
\end{frame}

\begin{frame}
    \frametitle{Preliminary results}
    \begin{enumerate}
        \item We can already create an immersive experience by superposing the original signal with the same one delayed.
        \item We can generate an audio file with respect to the direction we set
        \item For the rest of the project, our aim is to merge the two points above...
        \item ... and play with some filters a little more to enhance the listening experience.
    \end{enumerate}
\end{frame}

\begin{frame}
    \frametitle{The End}
    Thank you for listening. Now, questions (if any).
\end{frame}

\end{document}